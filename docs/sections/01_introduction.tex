\section{Einleitung}
Diese wissenschaftliche Ausarbeitung wird im Rahmen der Hochschulveranstaltung
\textit{Hot-Topics in der IT-Security} angefertigt und handelt über den
sicheren Datenaustausch zwischen Internet-of-Things-Geräten.  Besonders wird
hierbei auf die Attribute Based Encryption (ABE) eingegangen, welche ein
relativ junges Forschungsgebiet der Kryptographie darstellt. Es existieren
zwar einige ABE-Schemata, diese sind jedoch für leistungsschwache Geräte eher
ungeeignet.

Zu Beginn widmet sich die Ausarbeitung den grundlegenden mathematischen
Hintergründen, die benötigt werden, um die hier vorgestellten Schemata verstehen
zu können. Dazu gehört vor allem die Definitionen von Zugriffsregeln bzw.
-strukturen. In der Praxis verwendet man beispielsweise gerne Zugriffsstrukturen
in Form eines binären Baumes, um eine oder mehrere Zugriffsregeln zu definieren.
In den kryptografischen Algorithmen findet man jedoch häufig die Beschreibung
von Zugriffsregeln durch sogenannte \textit{Linear-Secret-Sharing-Schemes} in
Form einer Matrix. Dies stellt vor allem eine sehr viel elegantere Darstellung
von Zugriffsregeln dar, als binäre Bäume. Da ein Baum jedoch wesentlich besser
nachvollzogen werden kann, wird eine Zugriffsstruktur erst in Form eines Baumes
definiert und dann später in eine Matrix umgewandelt.

Diese Ausarbeitung beschäftigt sich hauptsächlich mit der Arbeit von
\cite{phoabe}, welche ein neues attribut-basiertes Verschlüsselungsschema
vorstellt: \textit{PHOABE}. Dieses Schema widmet sich bekannten Problemen der
Attribut-basierten Verschlüsselung, wie zum Beispiel das Verstecken der
Zugriffsregelungen (\textit{Access-Policies}), die sensitive Daten enthalten
können. Zusätzlich fordert das Schema eine Auslagerung des
Entschlüsselungsvorganges, welches aufgrund von bilinearen Abbildungen relativ
rechenintensiv ist. Hierbei sollen sogenannte \textit{Semi-trusted Cloud
Server} den aufwändigen Teil der Berechnung über\-nehmen, damit auch
leistungsschwache Geräte solche attribut-basierte Ver\-schlüs\-sel\-ung\-en
durchführen können.
