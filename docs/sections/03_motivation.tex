\section{Motivation}
Die attribut-basierte Verschlüsselung ist, wie in der Problembeschreibung
bereits erwähnt, ineffizienter, je mehr Attribute verwendet werden. Besonders
die Entschlüsselung wird durch die bilinearen Operationen extrem
rechenaufwändig, was zur Folge hat, dass dieses Schema auf ressourcenärmeren
Geräten und damit auf IoT nur bedingt anwendbar ist. Eine wichtige Frage, die
sich also stellt ist, wie man ABE für eben solche Anwendungsfälle anwendbar
gestalten kann.

In weiterführenden Arbeiten, so auch der von \cite{green}, wird ein neues
Konzept in Verbindung mit attribut-basierter Verschlüsselung vorgestellt, um
diese effizienter zu gestalten. Für die kostspielige Entschlüsselung ist nun
nicht mehr alleine der jeweilige Benutzer zuständig, sondern eine weitere
Instanz. Ein halb vertrauenswürdiger Cloud-Server (\textit{semi-trusted cloud
server}, STCS), welcher den rechenaufwändigen Teil der Entschlüsselung
übernehmen soll. Halb vertrauenswürdig bedeutet, dass wir dem Server
vertrauen, dass dieser uns die gewünschten Resultate liefert, jedoch selbst
versucht private Daten auszulesen. Der STCS soll lediglich einen Teil der
Entschlüsselung übernehmen. Anschließend wird das Ergebnis vom Benutzer selbst
entschlüsselt. So soll garantiert werden, dass die rechenaufwändige Arbeit vom
STCS und nicht vom Benutzer übernommen wird und trotzdem keinerlei
Informationen über die eigentliche Nachricht an dem Server preisgegeben wird.

Eine weitere Überlegung von \cite{phoabe} ist folgende. Was ist, wenn der STCS
das teilweise entschlüsselte Chiffrat fälscht? Es muss also zusätzlich die
Möglichkeit bestehen, dass das Ergebnis der Entschlüsselung des STCS'
überprüft werden kann. Einige vergangene Arbeiten haben sich dem Problem
gewidmet, jedoch keine effiziente Lösung geboten oder sind inpraktikabel für
IoT, da sie sich auf einer einzigen Autorität stützen. Da IoT jedoch
hauptsächlich über mehrere Domänen kommuniziert, ist dies keine praktikable
Lösung. Die Motivation hinter \cite{phoabe} ist also, ein ausgelagertes,
verifizierbares multi-authority ABE-Schema zu entwerfen.
