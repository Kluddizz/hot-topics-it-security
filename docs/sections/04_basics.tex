\section{Grundlagen}
\subsection{Zugriffsregeln}
Grundsätzlich werden Regeln für den Datenzugriff durch zwei Formate
repräsentiert. Zum Einen durch boolsche Funktionen und zum Anderen durch
sogenannte Linear Secret Sharing Schemes (LSSS). In diesem Abschnitt sollen
beide Repräsentationsmöglichkeiten eingeführt werden.

\begin{definition}[Zugriffsstrukturen \cite{abe}]\label{def:access-structures}
	Sei $\left\{ P_1, P_2, ..., P_n \right\}$ die Menge der beteiligten
	Parteien. Eine Sammlung $\mathbb{A} \subseteq 2^{\left\{ P_1, P_2, ..., P_n
	\right\}}$ ist monoton, wenn $\forall B, C .\;\; B \in \mathbb{A}
	\;\;\land\;\; B \subseteq C \implies C \in \mathbb{A}$. Eine
	Zugriffsstruktur ist damit eine Sammlung $\mathbb{A}$ von nicht-leeren
	Untermengen von dem Universum $\left\{ P_1, P_2, ... P_n \right\}$. Alle
	Mengen $A \in \mathbb{A}$ werden als authorisierte Mengen während
	die nicht in $\mathbb{A}$ vertetenden Mengen als unauthorisierte
	bezeichnet werden.
\end{definition}

Definition ~\ref{def:access-structures} kann dabei so interpretiert werden,
als dass alle Obermengen jedes Elements $B \in \mathbb{A}$ verteten sein
müssen. Im Folgenden soll ein Beispiel dies verdeutlichen.

\begin{example}
	Sei $\mathbb{U} = \left\{ 1, 2, 3, 4 \right\}$ ein Universum und $\mathbb{A}
	\subseteq 2^\mathbb{U}$ eine Zugriffsstruktur.

	Die Zugriffsstruktur $\mathbb{A} = \left\{ \left\{1,2\right\},
	\left\{3,4\right\} \right\}$ ist nicht monoton, da das Element
	$\left\{1,2,3\right\}$ nicht vorhanden ist.

	Die Zugriffsstruktur $\mathbb{A} = \left\{ \left\{3,4\right\},
	\left\{1,3,4\right\}, \left\{1,2,3,4\right\} \right\}$ ist monoton, da alle
	Obermengen der Elemente von $\mathbb{A}$ vorhanden sind.
\end{example}

\begin{definition}[Linear Secret-Sharing Schemes \cite{abe}]
	Ein Linear Secret-Sharing Scheme (LSSS) über eine Menge von Parteien $P$ ist
	linear, wenn

	\begin{enumerate}
		\item Die Anteile (shares) für jede Partei einen Vektor $\vec{v} \in
			\mathbb{Z}_p^{n+1}$ formen.
		\item Eine Matrix $M$ existiert, die $l$ Zeilen und $n+1$ Spalten enthält.
			Jede $i$-te Zeile aus $M$ mit $i \in \left\{1, ..., l\right\}$ ist dann
			mit einer Partei $x_i \in P$ beschriftet. Wenn wir den Spaltenvektor $v =
			\left(s, r_1, r_2, ..., r_n\right)$ betrachten, wobei $s \in \mathbb{Z}_p$ das zu
			teilende Geheimnis ist und $r_1, ..., r_n \in \mathbb{Z}_p$ zufällig
			gewählt werden, dann ist $Mv$ ein Vektor bestehend aus $l$ Anteilen des
			Geheimisses $s$.
	\end{enumerate}
\end{definition}

\begin{example}
	Gegeben sei eine LSSS-Matrix $M \in \mathbb{Z}_2^{l \times n}$ und die Abbilding $\rho : \mathbb{N^+}
	\times \mathbb{Z}_p^{l \times n} \to \mathbb{Z}_p^n$, welche die
	$i$-te Zeile der Matrix $M$ zurückgibt. Zudem werden den Zeilen Parteien
	$P_i$ zugewiesen, sodass $\rho(i, M)$ den Anteil der jeweiligen Partei liefert.
	$$M = \kbordermatrix{
				&   &   &   \\
		P_2 &1 & 1 & 0 & 1 \\
		P_2 &	0 & 1 & 1 & 0 \\
		P_1 &	0 & 1 & 1 & 0 \\
		P_3 &	1 & 1 & 0 & 0 \\
		P_4 &	0 & 0 & 1 & 1
		}$$

	Betrachten wir die Kombination $\left\{ P_2, P_4 \right\}$. Mithilfe der
	Abbildung $\rho$ erhalten wir die dementsprechenden Zeilen.

	\begin{align*}
		\rho\left(0, M\right) = \vec{v_1}_{P_2} = \left(1, 1, 0, 1\right) \\
		\rho\left(1, M\right) = \vec{v_2}_{P_2} = \left(0, 1, 1, 0\right) \\
		\rho\left(4, M\right) = \vec{v}_{P_4} = \left(0, 0, 1, 1\right)
	\end{align*}

	Um die Authentizität zu prüfen, muss die Summe der Zeilenvektoren $e =
	\left(1, 0, 0, 0\right)$ ergeben. Wir berechnen nun also die Summe der
	Vektoren
	
	$$\vec{v_1}_{P_2} + \vec{v_2}_{P_2} + \vec{v}_{P_4} = (1, 2, 2, 2) \equiv
	(1, 0, 0, 0) \mod 2$$

	und sehen, dass die Kombination $\left\{P_2, P_4\right\}$ authorisiert ist.
\end{example}

\subsection{Bilineare Abbildungen}
\begin{definition}[Bilineare Abbildungen \cite{abe}]
	Sei $\mathbb{P}$ die Menge aller Primzahlen und $\mathbb{G}$, $\mathbb{G}_T$
	zwei multiplikative zyklische Gruppen mit einer Ordnung $p \in \mathbb{P}$.
	Sei $g$ ein Generator von $\mathbb{G}$ und $e: \mathbb{G} \times \mathbb{G}
	\to \mathbb{G}_T$ eine bilineare Abbildung mit folgenden Eigenschaften.

	\begin{enumerate}
		\item Bilinearität: $\forall u, v \in \mathbb{G},\;\; \forall a, b \in
			\mathbb{Z}_p. \;\; e(u^a, v^b) = e(u, v)^{ab}$
		\item Nicht-Entartung: $e(g, g) \neq 1$
		\item Effizient Berechenbar: Die Gruppenoperation von $\mathbb{G}$ und die
			bilineare Abbildung $e$ sind effizient berechenbar.
	\end{enumerate}
\end{definition}

\begin{example}
	Sei $(G, +)$ eine Gruppe und $x, y, z \in G$. Sei $(G_T, *)$ eine multiplikative
	Gruppe und $e: G \times G \to G_T$ eine bilineare Abbilding. Dann gilt

	$e(3x, y) = e(x+x+x, y) = e(x, y) * e(x, y) * e(x, y) = e(x, y)^3 = e(x,
	3y)$.
\end{example}
