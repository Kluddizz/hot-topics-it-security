\newcommand{\sizeof}[1]{\texttt{size}\left( #1 \right)}
\renewcommand{\O}[1]{\mathcal{O} \left( #1 \right)}

\newpage
\section{Performance Analyse}
Im Folgenden wird der Speicher- und Ressourcenverbrauch vom PHOABE-Schema
diskutiert bzw. analysiert. Hierfür werden die in \cite{phoabe} definierten
Symbole verwendet, um dies darzustellen.

\begin{itemize}
  \item $E_1$ stellt eine Potenzierung in $\mathbb{G}_1$ dar.
  \item $E_T$ stellt eine Potenzierung in $\mathbb{G}_T$ dar.
  \item $\tau_P$ stellt eine Ausführung der Pairing-Funktion $e$ dar.
  \item $n$ stellt die Anzahl aller Attribute einer Zugriffsstruktur dar.
  \item $\lvert S \rvert$ stellt die Anzahl aller Attribute dar, die einem
    Benutzer zugeordnet wurden.
\end{itemize}

\paragraph{Schlüsselgröße.}
Laut \cite{phoabe} beträgt die Speicherkapazität für einen Schlüssel eines
Benutzers $2 \lvert S \rvert$, wobei $\lvert S \rvert$ die Anzahl der für
den Benutzer zugewiesenen Attribute darstellt. Um dies nachzuvollziehen, müssen
wir uns den Keygen-Algorithmus einmal näher anschauen. Im ersten Schritt
berechnet dieser für jedes Attribut $i \in S$ einen Wert $K_{1,i}$ und einen Wert
$K_{2,i}$, wobei $S$ die Menge aller Attribute darstellt, die dem Benutzer
zugewiesen wurden. Da sich der Schlüssel aus den beiden resultierenden Mengen
zusammensetzt, besitzt dieser die Größe $\lvert S \rvert + \lvert S \rvert = 2
\lvert S \rvert$.

\paragraph{Größe des Transformationsschlüssels.}
Die Größe eines Transformationsschlüssels beträgt $\lvert S \rvert + 3$. Auch
hier schauen wir genauer auf den Transform-Algoritmus, um dies nachzuvollziehen.
Zuerst wird festgestellt, welche Attribute für eine Entschlüsselung benötigt
werden. Dies entspricht der Menge $S$, die alle Attribute für den Benutzer
enthält. Zudem wird ein zufälliges $z \in \mathbb{Z}_p$ gewählt, welches
einen zusätzlichen Teil des Transformationsschlüssels darstellt. Anschließend
wird für jedes Attribut $i \in S$ der Wert $K_{1,i}^{1/z}$ berechnet. Als
weitere Teile des Schlüssels werden die beiden Werte $g^{1/z}$ und
$H(GID)^{1/z}$ berechnet. Die Größe des gesamten Schlüssels beträgt damit $1 +
\lvert S \rvert + 1 + 1 = \lvert S \rvert + 3$.

\paragraph{Geheimtextgröße.}
Im Encrypt-Algorithmus wird der Geheimtext aus insgesamt drei Teilen gebildet.
Die Größe $\sizeof{CT}$ bildet sich daher ebenfalls aus genau drei Teilen und lässt
sich als
\begin{align*}
	\sizeof{CT} = \sizeof{CT_{ABE}} + \sizeof{CT_{sym}} + \sizeof{V}
\end{align*}
beschreiben. Dabei stellt $CT_{ABE} = \left( h, C_0, C_{1,i}, C_{2,i},
C_{3,i} \right)$ den at\-tri\-but-ba\-sier\-ten Teil und $CT_{sym} =
\text{encrypt}_{sym}\left(K_{sym}, m\right)$ die symmetrisch ver\-schlüs\-selte
Nachricht dar. Außerdem gibt der Algorithmus einen Verifizierungsschlüssel $V$
zurück. Die Größe $\sizeof{V}$ beträgt damit $1$. Um $\sizeof{CT_{ABE}}$ zu
bestimmen, müssen wir uns die einzelnen Komponenten genauer anschauen. Die
Komponenten $h$ und $C_0$ benötigen jeweils einen Speicherplatz während
$C_{1,i}$, $C_{2,i}$ und $C_{3,i}$ für jedes Attribut innerhalb der
Zugriffsstruktur berechnet werden und damit insgesamt drei mal der Anzahl an
Attributen groß sind. Die Größe $\sizeof{CT_{ABE}}$ beträgt somit $2 + 3n$,
wobei $n$ die Anzahl aller Attribute, die in dem Schema vorkommen, darstellt.
Zuletzt muss noch die Größe von $\sizeof{CT_{sym}}$ bestimmt werden. Da
$CT_{sym}$ lediglich die symmetrische Verschlüsselung einer Nachricht darstellt,
beträgt der benötigte Speicherplatz zum Speichern dieser Variable $1$. Der
gesamte Geheimtext, der sich aus all den Teilen zusammensetzt, benötigt daher
$\sizeof{CT} = 2 + 3n + 1 + 1 = 4 + 3n$ Speichereinheiten.

\paragraph{Kosten der Verschlüsselung.}
Es soll nun auf die Komplexität der Verschlüsselung einer Nachricht eingegangen
werden. Beginnen wir mit der Berechnung aller $q_i = e\left((g^y)^x,
H'(a_i)\right)$ zum Rekonstruieren der LSSS-Matrix. Hier wird für jedes $q_i$
eine Potenzierung in $\mathbb{G}_1$ und einmal der Hashfunktion über $a_i$
ausgeführt. Da für jedes Attribut $a_i$ innerhalb der Zugriffsstruktur ein $q_i$
berechnet wird, ergibt sich folgende Komplexität für diesen Teil: $n \cdot E_1 +
n \cdot \mathcal{O}(H) + n \cdot \tau_P = n (E_1 + \mathcal{O}(H) + \tau_P)$.
Darauf folgend wird $CT_{ABE} = (h, C_0, C_{1,i}, C_{2,i}, C_{3,i})$ berechnet.
Die Komplexität von $CT_{ABE}$ wird also durch seine Komponenten bestimmt, die
nun genauer untersucht werden.
\begin{align*}
  & h = g^a & \implies E_1 \\
  & C_0 = R \cdot e(g, g)^s & \implies E_T \\
  & C_{1,i} = g^{\lambda_{\rho(i)}}g^{\alpha_{\rho(i)}p_i} & \implies n \cdot 2E_1 \\
  & C_{2,i} = g^{p_i} & \implies n \cdot E_1 \\
  & C_{3,i} = g^{t_{\rho(i)}p_i}g^{\omega_i} & \implies n \cdot 2E_1
\end{align*}
Zudem wird eine Hashfunktion $H$ für die Berechnung von $R_0$, $K_{sym}$ und $V$
ausgeführt. Insgesamt erhalten wir damit eine Komplexität von
\begin{align*}
  \O{encrypt} &= \O{\{ q_i \;\vert\; 0 \leq i < n \}} + \O{CT_{ABE}} + 3\O{H} \\
  &= n (E_1 + \mathcal{O}(H) + \tau_P) + E_1 + 5n \cdot E_1
  + E_T + 3\mathcal{O}(H).
\end{align*}
Da $h$ und alle $q_i$ lediglich einmal initial berechnet werden,
müssen diese nicht bei jeder Verschlüsselung erneut bestimmt werden. Der Aufwand
der Berechnungen beträgt damit $\mathcal{O}(encrypt) = 5n \cdot E_1 + E_T +
3\mathcal{O}(H)$.

\paragraph{Kosten der Entschlüsselung (Benutzer).}
Es wird nun ein entscheidener Algorithmus des PHOABE-Schemas analysiert, der die
Lauffähigkeit auf ressourcenarmen Geräten sicherstellen soll. In den meisten
attribut-basierten Schemata ist der Decrypt-Algorithmus besonders in Bezug zu
IoT unpraktikabel, da dieser zu viel Rechenleistung erfordert \cite{phoabe}. Es
wird also nun analysiert, wie aufwändig dieser Algorithmus ist. Zu Beginn wird
die bereits teilweise vom STCS entschlüsselte Nachricht $m'$ verwendet, um einen
Wert $R = \frac{C_0}{e(g, g)^s}$ zu berechnen. Dies entspricht einer
Potenzierung in $\mathbb{G}_T$. Anschließend wird mithilfe von Hashfunktionen
ein Wert $R_0 = H_0(R)$ berechnet und validiert, ob $H_2(R_0 \;||\; CT_{sym}$
gleich $V$ ist. Falls ja, wird der symmetrische Schlüssel $K_{sym} = H_1(R)$
berechnet und zum Entschlüsseln der Nachricht $m = decrypt_{sym}(K_{sym},
CT_{sym})$ verwendet. Insgesamt werden also eine Potentierung in
$\mathbb{G}_T$ und drei Ausführungen von Hashfunktionen durchgeführt.
Die Komplexität dieses Algorithmus' ist damit $\O{decrypt} = E_T + 3\O{H}$.

\paragraph{Kosten der Entschlüsselung (STCS).}
Nun soll der Aufwand für den STCS analysiert werden. Dieser hat die Aufgabe eine
verschlüsselte Nachricht teilweise zu entschlüsseln, um den Rechenaufwand für
den Endnutzer so gering wie möglich zu halten. Hierfür berechnet dieser für jede
Zeile der LSSS-Matrix einen Wert
\begin{align*}
  R_i = \frac{e(g^{1/z}, C_{1,i}) \cdot e(H'(GID)^{1/z}, C_{3,i})}{e(g^{\alpha_i
  / z} \cdot H'(GID)^{t_i/z}, C_{2, i})}.
\end{align*}
Hierbei wird ein Aufwand von $3\tau_P + 2E_1 + 2\O{H}$ erzeugt. Jedoch können
die Potenzierungen in $\mathbb{G}_1$ und die Berechnungen der Hashwerte vorher
berechnet und dann wiederverwendet werden. Dies reduziert den Aufwand auf
$3\tau_P$. Da dies für $n$ Attribute durchgeführt werden muss, beträgt der
Aufwand für diesen Schritt $3n\tau_P$. Anschließend wird die Ausgabe $m' =
e(g, g)^{\frac{s}{z}}$ berechnet, welches zu einem Gesamtaufwand von
$\O{decrypt_{out}} = 3n\tau_P + E_T$ führt.
