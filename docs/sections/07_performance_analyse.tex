\newpage
\section{Performance Analyse}
\paragraph{Schlüsselgröße.}
Laut \cite{phoabe} beträgt die Speicherkapazität für einen Schlüssel eines
Benutzers $2 \lvert S \rvert$, wobei $\lvert S \rvert$ die Anzahl der für
den Benutzer zugewiesenen Attribute darstellt. Um dies nachzuvollziehen, müssen
wir unsschauen den Keygen-Algorithmus einmal näher anschauen. Im ersten Schritt
berechnet dieser für jedes Attribut $i \in S$ einen Wert $K_{1,i}$ und einen Wert
$K_{2,i}$, wobei $S$ die Menge aller Attribute darstellt, die dem Benutzer
zugewiesen wurden. Da sich der Schlüssel aus den beiden resultierenden Mengen
zusammensetzt, besitzt dieser die Größe $\lvert S \rvert + \lvert S \rvert = 2
\lvert S \rvert$.

\paragraph{Größe des Transformationsschlüssels.}
Die Größe eines Transformationsschlüssels beträgt $2 \lvert S \rvert + 3$. Auch
hier schauen wir genauer auf den Transform-Algoritmus, um dies nachzuvollziehen.
Zuerst wird festgestellt, welche Attribute für eine Entschlüsselung benötigt
werden. Dies entspricht der Menge $S$, die alle Attribute für den Benutzer
enthält. Zudem wird ein zufälliges $z \in \mathbb{Z}_p$ gewählt, welches
einen zusätzlichen Teil des Transformationsschlüssels darstellt. Anschließend
wird für jedes Attribut $i \in S$ der Wert $K_{1,i}^{1/z}$ berechnet. Als
weitere Teile des Schlüssels werden die beiden Werte $g^{1/z}$ und
$H(GID)^{1/z}$ berechnet. Die Größe des gesamten Schlüssels beträgt damit $1 +
\lvert S \rvert + 1 + 1 = \lvert S \rvert + 3$.

\paragraph{Geheimtextgröße.}
Im Encrypt-Algorithmus wird der Geheimtext aus insgesamt drei Teilen gebildet.
Die Größe dessen bildet sich daher ebenfalls aus genau drei Teilen und lässt
sich als $N_{CT} = \text{size} (CT_{ABE}) + \text{size} (CT_{sym}) + \text{size}
(V)$ beschreiben. Dabei stellt $CT_{ABE} = \left( h, C_0, C_{1,i}, C_{2,i},
C_{3,i} \right)$ den at\-tri\-but-basierten Teil und $CT_{sym} =
\text{encrypt}_{sym}\left(K_{sym}, m\right)$ die symmetrisch ver\-schlüs\-selte
Nachricht dar.

\paragraph{Kosten der Verschlüsselung.}
\paragraph{Kosten der Entschlüsselung (Benutzer).}
\paragraph{Kosten der Entschlüsselung (STCS).}
