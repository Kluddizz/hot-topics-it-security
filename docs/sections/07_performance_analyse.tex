\newcommand{\sizeof}[1]{\small\texttt{size}\left( #1 \right)}
\renewcommand{\O}[1]{\mathcal{O} \left( #1 \right)}

\newpage
\section{Performance Analyse}
\paragraph{Schlüsselgröße.}
Laut \cite{phoabe} beträgt die Speicherkapazität für einen Schlüssel eines
Benutzers $2 \lvert S \rvert$, wobei $\lvert S \rvert$ die Anzahl der für
den Benutzer zugewiesenen Attribute darstellt. Um dies nachzuvollziehen, müssen
wir unsschauen den Keygen-Algorithmus einmal näher anschauen. Im ersten Schritt
berechnet dieser für jedes Attribut $i \in S$ einen Wert $K_{1,i}$ und einen Wert
$K_{2,i}$, wobei $S$ die Menge aller Attribute darstellt, die dem Benutzer
zugewiesen wurden. Da sich der Schlüssel aus den beiden resultierenden Mengen
zusammensetzt, besitzt dieser die Größe $\lvert S \rvert + \lvert S \rvert = 2
\lvert S \rvert$.

\paragraph{Größe des Transformationsschlüssels.}
Die Größe eines Transformationsschlüssels beträgt $2 \lvert S \rvert + 3$. Auch
hier schauen wir genauer auf den Transform-Algoritmus, um dies nachzuvollziehen.
Zuerst wird festgestellt, welche Attribute für eine Entschlüsselung benötigt
werden. Dies entspricht der Menge $S$, die alle Attribute für den Benutzer
enthält. Zudem wird ein zufälliges $z \in \mathbb{Z}_p$ gewählt, welches
einen zusätzlichen Teil des Transformationsschlüssels darstellt. Anschließend
wird für jedes Attribut $i \in S$ der Wert $K_{1,i}^{1/z}$ berechnet. Als
weitere Teile des Schlüssels werden die beiden Werte $g^{1/z}$ und
$H(GID)^{1/z}$ berechnet. Die Größe des gesamten Schlüssels beträgt damit $1 +
\lvert S \rvert + 1 + 1 = \lvert S \rvert + 3$.

\paragraph{Geheimtextgröße.}
Im Encrypt-Algorithmus wird der Geheimtext aus insgesamt drei Teilen gebildet.
Die Größe dessen bildet sich daher ebenfalls aus genau drei Teilen und lässt
sich als $\sizeof{CT} = \text{size} (CT_{ABE}) + \text{size} (CT_{sym}) + \text{size}
(V)$ beschreiben. Dabei stellt $CT_{ABE} = \left( h, C_0, C_{1,i}, C_{2,i},
C_{3,i} \right)$ den at\-tri\-but-basierten Teil und $CT_{sym} =
\text{encrypt}_{sym}\left(K_{sym}, m\right)$ die symmetrisch ver\-schlüs\-selte
Nachricht dar. Außerdem gibt der Algorithmus einen Verifizierungsschlüssel $V$
zurück. Die Größe $\sizeof{V}$ beträgt damit $1$. Um \\ $\sizeof{CT_{ABE}}$ zu
bestimmen, müssen wir uns die einzelnen Komponenten genauer anschauen. Die
Komponenten $h$ und $C_0$ benötigen jeweils einen Speicherplatz während
$C_{1,i}$, $C_{2,i}$ und $C_{3,i}$ für jedes Attribut innerhalb der
Zugriffsstruktur berechnet werden und damit insgesamt drei mal der Anzahl an
Attributen groß sind. Die Größe $\sizeof{CT_{ABE}}$ beträgt somit $2 + 3n$,
wobei $n$ die Anzahl aller Attribute, die in dem Schema vorkommen, darstellt.
Zuletzt muss noch die Größe von $\sizeof{CT_{sym}}$ bestimmt werden. Da
$CT_{sym}$ lediglich die symmetrische Verschlüsselung einer Nachricht darstellt,
beträgt der benötigte Speicherplatz zum Speichern dieser Variable $1$. Der
gesamte Geheimtext, der sich aus all den Teilen zusammensetzt, benötigt daher
$\sizeof{CT} = 2 + 3n + 1 + 1 = 4 + 3n$ Speichereinheiten.

\paragraph{Kosten der Verschlüsselung.}
Es soll nun auf die Komplexität der Verschlüsselung einer Nachricht eingegangen
werden. Beginnen wir mit der Berechnung aller $q_i = e\left((g^y)^x,
H'(a_i)\right)$ zum Rekonstruieren der LSSS-Matrix. Hier wird für jedes $q_i$
eine Potenzierung in $\mathbb{G}_1$ und einmal der Hashfunktion über $a_i$
ausgeführt. Da für jedes Attribut $a_i$ innerhalb der Zugriffsstruktur ein $q_i$
berechnet wird, ergibt sich folgende Komplexität für diesen Teil: $n \cdot E_1 +
n \cdot \mathcal{O}(H) + n \cdot \tau_P = n (E_1 + \mathcal{O}(H) + \tau_P)$.
Darauf folgend wird $CT_{ABE} = (h, C_0, C_{1,i}, C_{2,i}, C_{3,i})$ berechnet.
Die Komplexität von $CT_{ABE}$ wird also durch seine Komponenten bestimmt, die
nun genauer untersucht werden.
\begin{align*}
  & h = g^a & \implies E_1 \\
  & C_0 = R \cdot e(g, g)^s & \implies E_T \\
  & C_{1,i} = g^{\lambda_{\rho(i)}}g^{\alpha_{\rho(i)}p_i} & \implies n \cdot 2E_1 \\
  & C_{2,i} = g^{p_i} & \implies n \cdot E_1 \\
  & C_{3,i} = g^{t_{\rho(i)}p_i}g^{\omega_i} & \implies n \cdot 2E_1
\end{align*}
Zudem wird eine Hashfunktion $H$ für die Berechnung von $R_0$, $K_{sym}$ und $V$
ausgeführt. Insgesamt erhalten wir damit eine Komplexität von
\begin{align*}
  \O{encrypt} &= \O{\{ q_i \;\vert\; 0 \leq i < n \}} + \O{CT_{ABE}} + 3\O{H} \\
  &= n (E_1 + \mathcal{O}(H) + \tau_P) + E_1 + 5n \cdot E_1
  + E_T + 3\mathcal{O}(H).
\end{align*}
Da $h$ und alle $q_i$ lediglich einmal initial berechnet werden,
müssen diese nicht bei jeder Verschlüsselung erneut bestimmt werden. Der Aufwand
der Berechnungen beträgt damit $\mathcal{O}(encrypt) = 5n \cdot E_1 + E_T +
3\mathcal{O}(H)$.

\paragraph{Kosten der Entschlüsselung (Benutzer).}


\paragraph{Kosten der Entschlüsselung (STCS).}
