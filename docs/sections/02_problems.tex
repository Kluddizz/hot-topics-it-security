\section{Problembeschreibung}
Das \textit{Internet der Dinge}, \textit{Internet of Things} (IoT), hat die
Aufgabe, verschiedenste Hardwarekomponenten miteinander zu verbinden und damit
einen Datenaustausch zu ermöglichen. Ein gutes Beispiel hierfür sind
Smartphones und Sensoren, die ständig miteinander kommunizieren müssen bzw.
sollen. Grundsätzlich können IoT-Anwendungen aufgrund ihrer
Kommunikationsfähigkeit in zwei Bereiche eingeteilt werden. In
Einzelanwendungen be\-nö\-ti\-gen man lediglich eine Autorität während es auch
Anwendungsfälle gibt, die über unterschiedliche Bereiche bzw. Domänen
kommunizieren müssen. Diese besitzen damit mehrere Autoritäten \cite{phoabe}.

Beide Klassen von IoT haben eins gemeinsam. Sie müssen einen sicheren Austausch
von Daten und die Sicherung der Privatsphäre gewährleisten. Hierfür werden
Daten verschlüsselt und mithilfe von Zugriffskontrollmechanismen nur
bestimmten Parteien zur Verfügung gestellt. Genau deshalb ist die
attribut-basierte Verschlüsselung ein attraktiver Kandidat, da sie
Zugriffskontrolle (\textit{Access-Control}) und Verschlüsselung miteinander
verknüpft. Ein Problem dabei exisitiert dennoch. Dieses Verfahren verwendet
bilineare Abbildungen, was vor allem in der Entschlüsselung viel Rechenaufwand
bedeutet. Für leistungsstarke Desktop-Rechner mag dies weniger eine Rolle
spielen, ist jedoch bezogen auf IoT ein großes Problem, denn dort
kommunizieren in der Regel Geräte mit sehr eingeschränkten Ressourcen
miteinander. Zudem erhöht sich der Rechenaufwand proportional zur Anzahl
der Attribute. Das Problem liegt also darin, Daten sicher und effizient
zwischen ressourcenarmen Geräten auszutauschen, ohne die Privatsphäre der
Benutzer zu verletzen.
