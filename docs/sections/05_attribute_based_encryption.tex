\section{Attribute Based Encryption (ABE)}
Im Gegensatz zur klassischen Public-Key-Verschlüsselung zielt die
attribut-basierte Verschlüsselung (ABE) darauf ab, ein Chiffrat für mehrere
anstatt für nur einen Benutzer zu erzeugen. Genauer ausgedrückt ist das Ziel
eines ABE-Schemas die Einhaltung der Anonymität eines Benutzers, da dieser nur
die benötigten Zugriffsbeschränkungen besitzen muss, um bestimmte Daten zu
entschlüsseln. Dafür werden die privaten Schlüssel und Chiffrate der Benuzer
(Parteien) als Zugriffsregeln bzw.  Attribute ausgelegt. Ferner ist ein Benutzer
in der Lage das Chiffrat mithilfe von Attributen oder festgelegten Regeln zu
entschlüsseln, sodass jeder Benutzer mit den richtigen Attributen Zugriff auf
die Daten hat. Die Attribute bzw.  Zugriffsregelungen werden dabei von einer
sogenannten Attribut-Autorität verteilt.  Grundsätzlich werden ABE-Schemata in
zwei unterschiedliche Kategorien eingeteilt
\cite{phoabe}.

\subsection{Key-Policy ABE}
KP-ABE-Schemata verwenden Zugriffsregeln als private Schlüs\-sel und berechnen
ein Chiffrat basierend auf ausgewählte Attribute. Als Beispiel soll die
Zugriffsregel $A \land B$ als Schlüssel dienen. Zudem wird ein Chiffrat mit
dem Attribut $\left\{A\right\}$ erzeugt. Versucht man nun das Chiffrat zu entschlüsseln,
so schlägt dies fehl, da die Zugriffsregel nicht erfüllt wird. Werden hingegen
die Attribute $\left\{A, B\right\}$ verwendet, wird die Regel eingehalten und
das Chiffrat kann entschlüsselt werden. Hieraus wird ein mögliches Problem
sichtbar, denn mit einem solchen Schema lässt sich nicht kontrollieren welche
Benutzer Zugriff besitzen sollen. Stattdessen erhalten alle Benutzer Zugriff,
denen eine entsprechende Zugriffsregel zugewiesen wurde \cite{cp-abe-ieee}.

Ein Key-Policy-ABE-Schema besteht aus insgesamt vier Algorithmen \cite{kp-abe}.

\begin{enumerate}
	\item $\textit{Setup}\left(n\right) \to \left(PK, SK_M\right)$: Dieser
		Algorithmus nimmt als Eingabe den Sicherheitsparameter $n$ und liefert als
		Ergebnis eine Menge bestehend aus öffentlichen Parametern $PK$, sowie
		einen Master-Secret-Key $SK_M$.
	\item $\textit{KeyGen}\left(\mathbb{A}, SK_M\right) \to SK$: Als Eingabe
		nimmt dieser Algorithmus die Zu\-griffs\-struktur $\mathbb{A}$ und den
		Master-Secret-Key $SK_M$. Es wird ein privater Schlüssel $SK$
		zurückgegeben.
	\item $Enc\left(m, A, PK\right) \to c$: Der Verschlüsselungsalgorithmus
		nimmt als Eingabe die Nachricht $m$, eine nicht-leere Menge von Attributen
		$A$ und die öffentlichen Parameter $PK$. Als Ausgabe wird ein Chiffrat $c$
		erzeugt.
	\item $Dec\left(c, SK\right) \to \left\{m, \bot\right\}$: Dieser Algorithmus
		nimmt als Eingabe ein Chiffrat $c$ und den privaten Schlüssel $SK$. Als
		Ausgabe wird die Nachricht $m$ oder ein Fehler (dargestellt als $\bot$)
		erzeugt.
\end{enumerate}

\subsection{Ciphertext-Policy ABE}
CP-ABE verwendet im Gegensatz zu KP-ABE Attribute als private Schlüssel
\cite{cp-abe-ieee}. Zudem werden Chiffrate mithilfe von Zugriffsregeln
erzeugt, also genau engegengetzt zu dem Vorgehen bei KP-ABE. Nehmen wir an,
wir erzeugen einen Schlüssel für Benutzer 1 bestehend aus den Attributen
$\left\{A, B\right\}$ und einen Schlüssel für Benutzer 2 bestehend aus den
Attributen $\left\{B\right\}$. Bei einem Chiffrat, welches unter der
Zugriffsregel $A \land B$ erstellt wurde, ist Benutzer 1 dazu in der Lage,
dieses zu entschlüsseln.  Benutzer 2 jedoch nicht. Bei diesem Verfahren ist
man also in der Lage zu entscheiden, welche Daten für welchen Benutzer zur
Verfügung stehen sollen bzw. welche Attribute vorausgesetzt werden, um Geheimtexte
zu entschlüsseln.

Ein Ciphertext-Policy ABE-Schema besteht aus insgesamt vier Algorithmen
\cite{cp-abe}.

\begin{enumerate}
	\item $\textit{Setup}\left(n\right) \to \left(PK, SK_M\right)$: Dieser
		Algorithmus nimmt als Eingabe den Sicherheitsparameter $n$ und liefert als
		Ergebnis eine Menge bestehend aus öffentlichen Parametern $PK$, sowie
		einen Master-Secret-Key $SK_M$.
	\item $\textit{KeyGen}\left(A, SK_M\right) \to SK$: Als Eingabe
		nimmt dieser Algorithmus eine Menge von Attributen $A$ und den
		Master-Secret-Key $SK_M$. Es wird ein privater Schlüssel $SK$
		zurückgegeben.
	\item $Enc\left(m, \mathbb{A}, PK\right) \to c$: Der
		Verschlüsselungsalgorithmus nimmt als Eingabe die Nachricht $m$, eine
		Zugriffsstruktur $\mathbb{A}$ und die öffentlichen Parameter $PK$.  Als
		Ausgabe wird ein Chiffrat $c$ erzeugt.
	\item $Dec\left(c, SK\right) \to \left\{m, \bot\right\}$: Dieser Algorithmus
		nimmt als Eingabe ein Chiffrat $c$ und den privaten Schlüssel $SK$. Als
		Ausgabe wird die Nachricht $m$ oder ein Fehler (dargestellt als $\bot$)
		erzeugt.
\end{enumerate}

\subsection{Single- und Multi-Authority}
Die Erzeugung der privaten Schlüssel kann in einem ABE-Schema auf zwei
verschiedenen Arten durchgeführt werden. Zum einen existiert das Modell einer
zentralen Autorität, welche für die Generierung der Schlüssel für alle
Benutzer zuständig ist. Man bezeichnet diese ABE-Schemata als
\textit{central}- bzw. \textit{single-autority}. Da in diesem Verfahren eine
einzelne Autorität die Schlüssel bereitstellt, muss dieser vertraut werden.
Dies bringt natürlich ein gewisses Risiko mit sich, denn eine zentrale
Autorität besitzt sämtliche Schlüssel der Parteien und kann dementsprechend
alle Geheimtexte entschlüsseln.

Um dieses Sicherheitsrisiko zu umgehen, werden sogenannte
\textit{multi-authority} Schemata eingesetzt. Anstatt einer zentralen erzeugen
und verwalten mehrere Autoritäten die privaten Schlüssel. Eine einzelne
Autorität ist damit nicht in der Lage einen Geheimtext zu entschlüsseln, da
ihr nicht alle Bestandteile des benötigten Schlüssels zur Verfügung stehen.
Zudem ist dieses Verfahren entlastend für sämtliche Autoritäten, da die
Verwaltung der Schlüssel durch mehrere Instanzen realisiert wird.
