\documentclass{hsflensburg}
\title{Wissenschaftliche Ausarbeitung}
\subtitle{PHOABE: Policy-Hidden Outsourced Attribute Based Encryption}

% Authors
\author{
	\name{Florian Hansen}\\
  \institution{Hochschule Flensburg}
}

% Packages
\usepackage[utf8]{inputenc}
\usepackage[ngerman]{babel}
\begin{document}
	\maketitle
	\begin{abstract}
	\end{abstract}

	\section{Einleitung}
	Diese wissenschaftliche Ausarbeitung wird im Rahmen der Hochschulveranstaltung
	\textit{Hot-Topics in der IT-Security} angefertigt und handelt über den
	sicheren Datenaustausch zwischen Internet-of-Things-Geräten.  Besonders wird
	hierbei auf die Attribute Based Encryption (ABE) eingegangen, welche ein
	relativ junges Forschungsgebiet der Kryptographie darstellt. Es existieren
	zwar einige ABE-Schemata, diese sind jedoch für leistungsschwache Geräte eher
	ungeeignet.

	Diese Ausarbeitung beschäftigt sich hauptsächlich mit der Arbeit von
	\cite{phoabe}, welche ein neues attribut-basiertes Verschlüsselungsschema
	vorstellt: \textit{PHOABE}. Dieses Schema widmet sich bekannten Problemen der
	Attribut-basierten Verschlüsselung, wie zum Beispiel das Verstecken der
	Zugriffsregelungen (\textit{Access-Policies}), die sensitive Daten enthalten
	können. Zusätzlich fordert das Schema eine Auslagerung des
	Entschlüsselungsvorganges, welches aufgrund von bilinearen Abbildungen relativ
	rechenintensiv ist. Hierbei sollen sogenannte \textit{Semi-trusted Cloud
	Server} den aufwändigen Teil der Berechnung über\-nehmen, damit auch
	leistungsschwache Geräte solche attribut-basierte Ver\-schlüs\-sel\-ung\-en
	durchführen können.

	\section{Problembeschreibung}
	Das \textit{Internet der Dinge}, \textit{Internet of Things} (IoT), hat die
	Aufgabe, verschiedenste Hardwarekomponenten miteinander zu verbinden und damit
	einen Datenaustausch zu ermöglichen. Ein gutes Beispiel hierfür sind
	Smartphones und Sensoren, die ständig miteinander kommunizieren müssen bzw.
	sollen. Grundsätzlich können IoT-Anwendungen aufgrund ihrer
	Kommunikationsfähigkeit in zwei Bereiche eingeteilt werden. In
	Einzelanwendungen be\-nö\-ti\-gen man lediglich eine Autorität während es auch
	Anwendungsfälle gibt, die über unterschiedliche Bereiche bzw. Domänen
	kommunizieren müssen. Diese besitzen damit mehrere Autoritäten.

	Beide Formen von IoT haben eins gemeinsam. Sie müssen einen sicheren Austausch
	von Daten und die Sicherung der Privatsphäre gewährleisten. Hierfür werden
	Daten verschlüsselt und mithilfe von Zugriffskontrollmechanismen nur
	bestimmten Autoritäten zur Verfügung gestellt. Genau deshalb ist die
	attribut-basierte Verschlüsselung ein attraktiver Kandidat, da sie
	Zugriffskontrolle (\textit{Access-Control}) und Verschlüsselung miteinander
	verknüpft. Ein Problem dabei exisitiert dennoch. Dieses Verfahren verwendet
	bilineare Abbildungen, was vor allem in der Entschlüsselung viel Rechenaufwand
	bedeutet. Für leistungsstarke Desktop-Rechner mag dies weniger eine Rolle
	spielen, ist jedoch bezogen auf IoT ein großes Problem, denn dort
	kommunizieren in der Regel Geräte mit sehr eingeschränkten Ressourcen
	miteinander. Zudem erhöht sich der Rechenaufwand proportional zur Anzahl
	der Attribute. Das Problem liegt also darin, Daten sicher und effizient
	zwischen ressourcenarmen Geräten auszutauschen, ohne die Privatsphäre der
	Benutzer zu verletzen.

	\section{Motivation}
	Die attribut-basierte Verschlüsselung ist, wie in der Problembeschreibung
	bereits erwähnt, ineffizienter, je mehr Attribute verwendet werden. Besonders
	die Entschlüsselung wird durch die bilinearen Operationen extrem
	rechenaufwändig, was zur Folge hat, dass dieses Schema auf ressourcenärmeren
	Geräten und damit auf IoT nur bedingt anwendbar ist. Eine wichtige Frage, die
	sich also stellt ist, wie man ABE für eben solche Anwendungsfälle anwendbar
	gestalten kann.

	In weiterführenden Arbeiten, so auch der von \cite{green}, wird ein neues
	Konzept in Verbindung mit attribut-basierter Verschlüsselung vorgestellt, um
	diese effizienter zu gestalten. Für die kostspielige Entschlüsselung ist nun
	nicht mehr alleine der Benutzer zuständig, sondern eine weitere Instanz. Ein
	halb vertrauenswürdiger Cloud-Server (\textit{semi-trusted cloud server},
	STCS), welcher den rechenaufwändigen Teil der Entschlüsselung übernehmen
	soll. Damit der Server keinerlei Informationen über den Entschlüsselten
	Ciphertext erhält, soll dieser diesen lediglich zum Teil entschlüsseln.
	Anschließend wird das Ergebnis vom Benutzer selbst entschlüsselt. So soll
	garantiert werden, dass die rechenaufwändige Arbeit vom STCS und nicht vom
	Benutzer übernommen wird und trotzdem keinerlei Informationen über die
	eigentliche Nachricht an dem Server preisgegeben wird.

	Eine weitere Überlegung von \cite{phoabe} ist folgende. Was ist, wenn der STCS
	das teilweise entschlüsselte Chiffrat fälscht? Es muss also zusätzlich die
	Möglichkeit bestehen, dass das Ergebnis der Entschlüsselung des STCS'
	überprüft werden kann. Einige vergangene Arbeiten haben sich dem Problem
	gewidmet, jedoch keine effiziente Lösung geboten oder sind inpraktikabel für
	IoT, da sie sich auf einer einzigen Autorität stützen. Da IoT jedoch
	hauptsächlich über mehrere Domänen kommuniziert, ist dies keine praktikable
	Lösung. Die Motivation hinter \cite{phoabe} ist also, ein ausgelagertes,
	verifizierbares multi-authority ABE-Schema zu entwerfen.
	
	\section{Grundlagen}
	\subsection{Access-Policies}
	\subsection{Bilineare Abbildungen}

	\section{Attribute Based Encryption (ABE)}
	\subsection{Key-Policy ABE}
	\subsection{Ciphertext-Policy ABE}
	\subsection{Single-Authority}
	\subsection{Multi-Authority}

	\section{Policy-Hidden Outsourced ABE}
	\section{Fazit und Ausblick}

	\bibliography{literatur}
	\bibliographystyle{abbrv}
\end{document}

